\documentclass[11pt]{article}
\usepackage{graphicx} % Required for inserting images
\usepackage{amsmath}
\usepackage{yhmath}
\usepackage{amsthm}
\usepackage{comment}
\usepackage{amssymb,geometry,parskip}
\usepackage[dvipsnames]{xcolor}
\usepackage{todonotes}
\usepackage{mathtools}
\usepackage{pgfplots}
\geometry{tmargin=1in, bmargin=1in, lmargin=1in, rmargin = 1in}
\newcommand{\gonzalo}[1]{{\color{blue} [GMK: {#1}]}}
\usepackage{amsthm}
\newtheorem{theorem}{Theorem}
\newtheorem{claim}{Claim}
\usepackage{tikz}
\usetikzlibrary{matrix}
\usepackage{bbm}
\usepackage{threeparttable}
\usepackage{ragged2e}
\usepackage{float}
\usepackage{physics}

\DeclareMathOperator*{\argmin}{arg\,min}
\DeclareMathOperator*{\argmax}{arg\,max}
\newcommand{\indep}{\perp \!\!\! \perp}

\newcommand{\conv}{\xrightarrow{\;\;\;}}

\newcommand{\convp}{\xrightarrow{{\scriptscriptstyle \mathrm{\;p\;}}}}

\newcommand{\convas}{\xrightarrow{{\scriptscriptstyle \mathrm{a.s.}}}}

\newcommand{\convms}{\xrightarrow{{\scriptscriptstyle \mathrm{m.s.}}}}

\newcommand{\convd}{\xrightarrow{{\scriptscriptstyle \mathrm{\;d\;}}}}

\newcommand{\PP}{\mathbb{P}}

\newcommand{\EE}{\mathbb{E}}
\newcommand{\VV}{\mathbb{V}}

\newcommand{\R}{\mathbb{R}}

\newcommand{\defeq}{\vcentcolon=}

\author{Vaidehi Parameswaran}
\date{\today}

\begin{document}


\section{Identification of a Cobb-Douglas Production Function}
\subsection*{(a)}
We have by definition that
\[y_t = f(k_t, l_t, m_t) + \omega_t + \epsilon_t\]
and assume, as in GNR, that $k_t$ and $l_t$ are dynamic but $m_t$ is freely adjustable in each period, conditional on $\omega_t$. 

Following GNR, we can write
\begin{align*}
    \mathbb{E}[y_t \mid \Gamma_t] & = \mathbb{E}[f(k_t, l_t, m_t)] + \mathbb{E}[\omega_t \mid \Gamma_t] + \mathbb{E}[\epsilon_t \mid \Gamma_t] \\
                        & = \mathbb{E}[f(k_t, l_t, m_t)] + \mathbb{E}[\delta_0 + \delta_1 \omega_{t-1} + \eta_t \mid \Gamma_t] + \mathbb{E}[\epsilon_t \mid \Gamma_t] \\
                        & = \mathbb{E}[f(k_t, l_t, m_t)] + \delta_0 + \delta_1 \mathbb{E}[\omega_{t-1} \mid \Gamma_t] \\
                        & = \alpha_k k_t + \alpha_l l_t + \alpha_m \mathbb{E}[m_t \mid \Gamma_t] + \delta_0 + \delta_1 h(\phi(k_{t-1}, l_{t-1}, m_{t-1}) + d_{t-1} - f(k_{t-1}, l_{t-1}, m_{t-1})),
\end{align*}
where we have rewritten $\omega_t$ using its definition and use the fact that $\eta_t$ and $\epsilon_t$ are exogenous.

\subsection*{(b)}
The firm's problem with respect to materials is
\[\max_{M_t} P_t \mathbb{E}[F(K_t, L_t, M_t) \exp(\omega_t + \epsilon_t) \mid \Gamma_t] - \rho_t M_t.\]
The first order condition is given by
\[P_t \pdv{M_t} F(K_t, L_t, M_t) \exp(\omega_t) \mathcal{E} = \rho_t.\]
Rearranging, we have
\[\alpha_m K_t^{\alpha_k} L_t^{\alpha_l} M_t^{\alpha_{m} - 1} = \frac{\rho_t}{P_t \mathcal{E}} \exp(-\omega_t).\]
Taking logs,
\begin{align*}
    (\alpha_m - 1) m_t & = \log(\frac{\rho_t}{P_t \mathcal{E}}) - \omega_t - \alpha_k k_t - \alpha_l l_t - \log(\alpha_m) \\
                       & = d_t - \omega_t - \alpha_k k_t - \alpha_l l_t - \log(\alpha_m).
\end{align*} 
Recall from above that
\begin{align*}
    \mathbb{E}[y_t \mid \Gamma_t] & = \alpha_k k_t + \alpha_l l_t + \alpha_m \mathbb{E}[m_t \mid \Gamma_t ] + \mathbb{E}[\omega_t \mid \Gamma_t] \\
                    & = \alpha_k k_t + \alpha_l l_t + \frac{\alpha_m}{\alpha_m - 1} \mathbb{E}[d_t - \omega_t - \alpha_k k_t - \alpha_l l_t - \log(\alpha_m) \mid \Gamma_t] + \mathbb{E}[\omega_t \mid \Gamma_t] \\
                    & = \left( 1 - \frac{\alpha_m}{\alpha_m - 1} \right)(\alpha_k k_t + \alpha_l l_t + \mathbb{E}[\omega_t \mid \Gamma_t]) + \frac{\alpha_m}{\alpha_m - 1} (d_t - \log(\alpha_m)) \\
                    & = \frac{1}{1 - \alpha_m} (\alpha_k k_t + \alpha_l l_t + \mathbb{E}[\omega_t \mid \Gamma_t]) + \frac{\alpha_m}{\alpha_m - 1} (d_t - \log(\alpha_m)) \\
                    & = \frac{1}{1 - \alpha_m} (\alpha_k k_t + \alpha_l l_t + h(\phi_{t-1} + d_{t-1} - f(k_{t-1}, l_{t-1}, m_{t-1})) + \frac{\alpha_m}{\alpha_m - 1} (d_t - \log(\alpha_m))
\end{align*}

\subsection*{(c)}
Recall that
\[\mathbb{E}[y_t \mid k_t, l_t, m_t] \equiv \phi(k_t, l_t, m_t) + d_t.\]
By the FOC on materials,
\[\mathbb{E}[y_t \mid k_t, l_t, m_t] = \mathbb{E}[\omega_t + f(k_t, l_t, m_t) \mid k_t, l_t, m_t] = \mathbb{E}[d_t + m_t - log(\alpha_m) \mid k_t, l_t, m_t]. \]
Thus, 
\[\phi_t = m_t - \log(\alpha_m).\]
Plugging this into the previous expression,
\[\mathbb{E}[y_t \mid \Gamma_t] = \frac{1}{1 - \alpha_m} \left[ \alpha_k k_t + \alpha_l l_t + \delta_0 + \delta_1 h(m_{t-1} - \log(\alpha_m) + d_{t-1} - f(k_{t-1}, l_{t-1}, m_{t-1}) \right]  + \frac{\alpha_m}{\alpha_m - 1} (d_t - \log(\alpha_m)).\]

\subsection*{(d)}
We assume that prices are fixed, so $d_t$ doesn't change. By the logic of GNR, the only variables that exogenously shifts $m_t$ is $\eta_t$, which is unobserved. Conditional on $\Gamma_t$, the expectation of $m_t$ is a constant. Because we have $\frac{1}{1 - \alpha_m}$ multiplying the other factor coefficients, these also cannot be identified. 

\subsection*{(e)}
We can use the firm's materials FOC to write
\[\mathbb{E}[Y_t / M_t \mid \Gamma_t] \alpha_m = \rho / P \implies \alpha_m = \frac{\rho}{P} \cdot \frac{1}{ \mathbb{E}\left[ \frac{Y_t}{M_t} \mid \Gamma_t\right] } .\]
So we can identify $\alpha_m$, since the right-hand side is just data. From this, we can identify the other factor coefficients, e.g. by the procedure in (d). Finally, we can back out $\hat{\omega}_t$ for each observation, and run an autogregression of $\hat{\omega}_t$ on $\hat{\omega}_{t-1}$. This will identify $\delta_0$ and $\delta_1$. 

\section{Estimation of Production Functions}

\subsection*{(a) Summary Statistics}

I haven't used Julia in the past to produce latex tables so forgive the somewhat shabby formatting of the tables. 

\begin{table}[ht]
    \centering
    \setlength{\tabcolsep}{4pt} % Adjust column spacing
\renewcommand{\arraystretch}{1.2}
    \begin{tabular}{|r|r|r|r|r|r|r|}
  \hline
  \textbf{Variable} & \textbf{Observations} & \textbf{Mean} & \textbf{Median} & \textbf{Std Dev} & \textbf{p25} & \textbf{p75} \\\hline
  Output & 11393 & 13.7903 & 13.6358 & 1.82123 & 12.2636 & 15.2589 \\
  Investment & 11393 & 8.77443 & 9.75418 & 4.18482 & 7.61776 & 11.7141 \\
  Capital & 11393 & 11.9133 & 11.8011 & 2.09978 & 10.1414 & 13.6564 \\
  Hours & 11393 & 4.90301 & 4.58619 & 1.39403 & 3.66766 & 6.11561 \\
  Intermediate C & 11393 & 13.2585 & 13.1634 & 1.95939 & 11.7206 & 14.8021 \\\hline
\end{tabular}

    \caption{Summary Statistics - Unbalanced Panel}
    \label{tab:summary_statistics1}
\end{table}

\begin{table}[ht]
    \centering
    \setlength{\tabcolsep}{2pt} % Adjust column spacing
\renewcommand{\arraystretch}{1.2}
    \begin{tabular}{rrrrrrrrrrrrrrrrrrr}
  \hline
  \textbf{year} & \textbf{ 1} & \textbf{ 2} & \textbf{ 3} & \textbf{ 4} & \textbf{ 5} & \textbf{ 6} & \textbf{ 7} & \textbf{ 8} & \textbf{ 9} & \textbf{ 10} & \textbf{ 11} & \textbf{ 12} & \textbf{ 13} & \textbf{ 14} & \textbf{ 15} & \textbf{ 16} & \textbf{ 17} & \textbf{ 18} \\
\hline
  1990 & 14 & 36 & 50 & 61 & 35 & 9 & 56 & 26 & 15 & 15 & 64 & 16 & 73 & 13 & 37 & 39 & 25 & 13 \\
  1991 & 18 & 60 & 68 & 102 & 53 & 10 & 92 & 43 & 23 & 22 & 97 & 23 & 112 & 29 & 54 & 71 & 46 & 20 \\
  1992 & 23 & 75 & 86 & 124 & 58 & 11 & 108 & 52 & 27 & 25 & 127 & 28 & 133 & 37 & 70 & 86 & 58 & 25 \\
  1993 & 24 & 74 & 84 & 112 & 63 & 12 & 99 & 56 & 25 & 28 & 136 & 30 & 126 & 36 & 74 & 84 & 65 & 25 \\
  1994 & 29 & 85 & 92 & 119 & 75 & 11 & 105 & 59 & 22 & 35 & 129 & 25 & 129 & 35 & 76 & 85 & 72 & 24 \\
  1995 & 32 & 86 & 92 & 123 & 80 & 9 & 106 & 59 & 24 & 40 & 128 & 25 & 131 & 32 & 68 & 86 & 77 & 24 \\
  1996 & 34 & 91 & 88 & 125 & 84 & 9 & 100 & 59 & 25 & 41 & 126 & 23 & 132 & 41 & 69 & 92 & 80 & 24 \\
  1997 & 35 & 91 & 88 & 163 & 93 & 14 & 118 & 70 & 30 & 44 & 135 & 19 & 139 & 50 & 86 & 103 & 95 & 29 \\
  1998 & 30 & 88 & 85 & 160 & 90 & 12 & 112 & 67 & 31 & 39 & 128 & 18 & 135 & 46 & 83 & 101 & 92 & 26 \\
  1999 & 29 & 71 & 71 & 137 & 76 & 9 & 91 & 58 & 27 & 27 & 107 & 15 & 121 & 32 & 72 & 81 & 84 & 22 \\\hline
\end{tabular}

    \caption{Industry Year Counts - Unbalanced Panel}
    \label{tab:ind_year1}
\end{table}


\begin{table}[ht]
    \centering
    \setlength{\tabcolsep}{1pt} % Adjust column spacing
\renewcommand{\arraystretch}{1.2}
    \begin{tabular}{rrrr}
  \hline
  \textbf{year} & \textbf{Zero Investment} & \textbf{Zero Labour} & \textbf{Zero Materials} \\
 \hline
  1990 & 74 & 0 & 0 \\
  1991 & 143 & 0 & 0 \\
  1992 & 189 & 0 & 0 \\
  1993 & 239 & 0 & 0 \\
  1994 & 215 & 0 & 0 \\
  1995 & 196 & 0 & 0 \\
  1996 & 179 & 0 & 0 \\
  1997 & 193 & 0 & 0 \\
  1998 & 137 & 0 & 0 \\
  1999 & 112 & 0 & 0 \\\hline
\end{tabular}

    \caption{Zero Counts - Unbalanced Panel}
    \label{tab:zeros1}
\end{table}

Investment has the highest variance in the data. 
There are no firms with zero labour or zero materials used. 
There seems to be significant entry and exit across industries and years. 

\subsection*{(b) Summary Statistics - Balanced Panel}

The count of observations is 2,470 and lower as expected given entry and exit. 
Means of relevant variables are higher implying that there is selection into the balanced panel, i.e. larger or ``better'' firms are more likely to stay in the sample throughout.
\begin{table}[ht]
    \centering
    \setlength{\tabcolsep}{4pt} % Adjust column spacing
\renewcommand{\arraystretch}{1.2}
    \begin{tabular}{|r|r|r|r|r|r|r|}
  \hline
  \textbf{Variable} & \textbf{Observations} & \textbf{Mean} & \textbf{Median} & \textbf{Std Dev} & \textbf{p25} & \textbf{p75} \\\hline
  Output & 11393 & 13.7903 & 13.6358 & 1.82123 & 12.2636 & 15.2589 \\
  Investment & 11393 & 8.77443 & 9.75418 & 4.18482 & 7.61776 & 11.7141 \\
  Capital & 11393 & 11.9133 & 11.8011 & 2.09978 & 10.1414 & 13.6564 \\
  Hours & 11393 & 4.90301 & 4.58619 & 1.39403 & 3.66766 & 6.11561 \\
  Intermediate C & 11393 & 13.2585 & 13.1634 & 1.95939 & 11.7206 & 14.8021 \\\hline
\end{tabular}

    \caption{Summary Statistics - Balanced Panel}
    \label{tab:summary_statistics2}
\end{table}

\begin{table}[ht]
    \centering
    \setlength{\tabcolsep}{2pt} % Adjust column spacing
\renewcommand{\arraystretch}{1.2}
    \begin{tabular}{rrrrrrrrrrrrrrrrrrr}
  \hline
  \textbf{year} & \textbf{ 1} & \textbf{ 2} & \textbf{ 3} & \textbf{ 4} & \textbf{ 5} & \textbf{ 6} & \textbf{ 7} & \textbf{ 8} & \textbf{ 9} & \textbf{ 10} & \textbf{ 11} & \textbf{ 12} & \textbf{ 13} & \textbf{ 14} & \textbf{ 15} & \textbf{ 16} & \textbf{ 17} & \textbf{ 18} \\
  1990 & 7 & 21 & 15 & 27 & 15 & 3 & 27 & 13 & 6 & 6 & 26 & 5 & 33 & 5 & 10 & 15 & 10 & 3 \\
  1991 & 6 & 21 & 14 & 26 & 15 & 3 & 27 & 14 & 6 & 6 & 26 & 5 & 33 & 5 & 10 & 16 & 11 & 3 \\
  1992 & 7 & 21 & 14 & 26 & 14 & 3 & 27 & 14 & 6 & 6 & 26 & 5 & 33 & 5 & 10 & 16 & 11 & 3 \\
  1993 & 7 & 21 & 14 & 26 & 14 & 3 & 27 & 14 & 6 & 6 & 26 & 5 & 33 & 5 & 10 & 16 & 11 & 3 \\
  1994 & 7 & 21 & 14 & 26 & 14 & 3 & 27 & 14 & 6 & 6 & 26 & 5 & 33 & 5 & 10 & 16 & 11 & 3 \\
  1995 & 7 & 21 & 14 & 27 & 14 & 3 & 28 & 13 & 6 & 6 & 26 & 5 & 32 & 5 & 10 & 16 & 11 & 3 \\
  1996 & 7 & 21 & 14 & 27 & 14 & 3 & 28 & 13 & 6 & 6 & 26 & 5 & 32 & 5 & 10 & 16 & 11 & 3 \\
  1997 & 6 & 21 & 14 & 28 & 14 & 3 & 28 & 13 & 6 & 6 & 26 & 5 & 32 & 5 & 10 & 16 & 11 & 3 \\
  1998 & 6 & 21 & 14 & 27 & 14 & 4 & 27 & 13 & 6 & 6 & 26 & 5 & 33 & 5 & 10 & 16 & 11 & 3 \\
  1999 & 6 & 21 & 14 & 27 & 13 & 4 & 27 & 13 & 7 & 6 & 26 & 5 & 33 & 5 & 10 & 16 & 11 & 3 \\\hline
\end{tabular}

    \caption{Industry Year Counts - Balanced Panel}
    \label{tab:ind_year2}
\end{table}


\begin{table}[ht]
    \centering
    \setlength{\tabcolsep}{1pt} % Adjust column spacing
\renewcommand{\arraystretch}{1.2}
    \begin{tabular}{rrrr}
  \hline
  \textbf{year} & \textbf{Zero Investment} & \textbf{Zero Labour} & \textbf{Zero Materials} \\  
  1990 & 19 & 0 & 0 \\
  1991 & 28 & 0 & 0 \\
  1992 & 29 & 0 & 0 \\
  1993 & 32 & 0 & 0 \\
  1994 & 29 & 0 & 0 \\
  1995 & 29 & 0 & 0 \\
  1996 & 27 & 0 & 0 \\
  1997 & 24 & 0 & 0 \\
  1998 & 16 & 0 & 0 \\
  1999 & 22 & 0 & 0 \\\hline
\end{tabular}

    \caption{Zero Counts - Balanced Panel}
    \label{tab:zeros2}
\end{table}

\clearpage
\subsection*{(c) Standard Regression Models - Balanced Panel}

I execute the standard models in julia. 
I am going to skip printing output in tables for this one because I don't have much experience printing these out of julia. 
But the Jupyter notebook prints the output. 

I choose industry no. 7.
From these models, we can learn that
the coefficient on materials is high consistently across models, implying a high elasticity of output with respect to materials.
Can't interpret the coefficients too well, the sign on capital seems to change. Negative coefficient on capital does not make sense to me.

\subsection*{(d) Standard Regression Models - Unbalanced Panel}

Using the balanced panel, the coefficient on materials is now lower. 
The coefficient on labour is larger now. 
Thus the balanced panel likely has firms with large material usage and smaller labour usage in otuput. 

\clearpage
\subsection*{(e) Estimation in Stata}

I did the parts with a Stata command in Stata. 
I didn't want to spend time figuring out how to do them in Julia. 
I attempted to code up OP myself but gave up midway for the sake of time. 

\begin{table}[ht]
    \centering
    \small
    \setlength{\tabcolsep}{2pt}
    {
\def\sym#1{\ifmmode^{#1}\else\(^{#1}\)\fi}
\begin{tabular}{l*{4}{c}}
\hline\hline
                    &\multicolumn{1}{c}{(1)}&\multicolumn{1}{c}{(2)}&\multicolumn{1}{c}{(3)}&\multicolumn{1}{c}{(4)}\\
                    &\multicolumn{1}{c}{AB}&\multicolumn{1}{c}{BB}&\multicolumn{1}{c}{OP}&\multicolumn{1}{c}{LP}\\
\hline
L.log of gross output&       0.328***&       0.279***&       0.279***&               \\
                    &     (0.078)   &     (0.069)   &     (0.069)   &               \\
[1em]
log of capital      &       0.072** &       0.023   &       0.023   &       0.158   \\
                    &     (0.035)   &     (0.027)   &     (0.027)   &     (0.108)   \\
[1em]
log of labor        &       0.157   &       0.026   &       0.026   &       0.277***\\
                    &     (0.109)   &     (0.081)   &     (0.081)   &     (0.032)   \\
[1em]
log of int consumption&       0.662***&       0.673***&       0.673***&               \\
                    &     (0.069)   &     (0.070)   &     (0.070)   &               \\
[1em]
Constant            &      -1.215   &       0.649   &       0.649   &               \\
                    &     (0.790)   &     (0.441)   &     (0.441)   &               \\
\hline
Observations        &         607   &         794   &         794   &         987   \\
\hline\hline
\multicolumn{5}{l}{\footnotesize Standard errors in parentheses}\\
\multicolumn{5}{l}{\footnotesize * p<0.10, ** p<0.05, *** p<0.01}\\
\end{tabular}
}

    \caption{Production Function Estimates}
    \label{tab:prod_est}
\end{table}

\subsection*{(e) ACF}
The coefficients I get are $\beta_k = 0.25$ and $\beta_l = 0.82$. 
I think covergence is failing and I can't successfully debug. 
But both grid-search and Nelder-Mead yield similar coefficients. 
And these estimates make sense in comparison to the estimates in ACF (2015). 
I bootstrap to get standard errors. 

\subsection*{(e) GNR}
I did first stage and gave up. Sorry!

\end{document}